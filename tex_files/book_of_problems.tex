\documentclass[14pt, A4]{extarticle}
\usepackage[utf8]{inputenc}
\usepackage{amsmath}
\usepackage{amsfonts}
\usepackage{tikz}
\usepackage{tkz-euclide}
\usepackage{graphicx}
\graphicspath{ {./images/} }

\usepackage{mathtools}
\DeclarePairedDelimiter\ceil{\lceil}{\rceil}
\DeclarePairedDelimiter\floor{\lfloor}{\rfloor}

\setlength{\parindent}{0pt}

\title{Book of Problems}
\author{SUMS}
\date{March 2021}

\begin{document}

\maketitle
\newpage

\tableofcontents
\newpage

\section{Preface}
This is the SUMS Maths Space Book of Problems. It will be a repository for problems donated by SUMS members.
\newpage

\section{March 2021}
\subsection{Density of Natural numbers mod $\pi$}
Consider the function $f:\mathbb{N} \rightarrow [0,\pi]$ defined $\forall n$ by $f(n) = n - \floor{\frac{n}{\pi}}n$.

Determine - with proof - if the following is true.\\

$\forall x \in [0, \pi] \epsilon > 0 \quad \exists n \in \mathbb{N} \ \textrm{s.t} \ |f(n) - x| < \epsilon$\\

You may use without proof the fact that $\pi$ is irrational

\subsection{A block containing primes}
Prove that there exists a block of $1000$ consecutive natural numbers of which exactly $10$ are prime

\subsection{Primes in the Fibonacci sequence}
The first few terms of the Fibonacci sequence are as follows\\
$F_{1} = 1$\\
$F_{2} = 1$\\
$F_{3} = 2$\\
$F_{4} = 3$\\
$F_{5} = 5$\\
$F_{6} = 8$\\
$F_{7} = 13$\\

Note that $F_{4} = 3$ is the $4$th number in the sequence ($4$ not a prime) but the number is $3$ (a prime number).\\

Show that this is the only case in the Fibonacci sequence where a prime number has a non-prime index

\subsection{Cyclogon conundrum}
Consider a $n$ sided convex polygon resting with two vertices on a line (as exemplified below).\\

IMAGE TO BE ADDED\\

Pick a vertex to be denoted $X$.\\
Consider the 'leftmost' of the two points lying on the line and imagine rotating the polygon anticlockwise around that vertex until the next vertex lies on the line.\\
Repeat this process with the new leftmost vertex on the line until the processes has been done $n$ times.\\

You should find that our original two vertices should be the two now lying on the line (again). What is the length of the path traced by $X$?

\subsection{Intermediate value property of derivatives}
Given a differentiatable function $f: [a,b] \rightarrow \mathbb{R}$ $(a < b)$

Is it true that given $d$ s.t $f'(a) < d < f'(b)$ there exists a $c \in [a,b]$ s.t $f'(c) = d$?

\subsection{Impossible position on a Rubik's cube}
Given a 3x3x3 Rubik's cube. Is there a position that can be reached by disassembling and reassembling the cube that isn't reachable by simply making moves from the starting position?


\subsection{Sequence}
Let $(t_{n})_{n}$ be a sequence of positive real numbers.\\

We define the sequence $(a_{n})_{n}$ as follows\\

$a_{0} = 1$ and $a_{n} = (\sum_{k=0}^{n-1}a_{k})t_{n}$\\

Can you find a general formula for $a_n$ in terms only of the other sequence of $(t_n)_{n}$?\\

Under what conditions on $(t_{n})_{n}$ does $(a_{n})_{n}$ converge?\\

Hence, or otherwise, show that the sum of all possible products of reciprocals of natural numbers less than or equal to $n$ is $n$.


\subsection{Show solution space of $n$th order linear differential equation has dimention $n$}
Consider the solution space $V$ to the differential equation\\

$a_{n}(x)f^{(n)}(x) + ... + a_{0}(x)f(x) = 0 \ \ \forall x \in \mathbb{R}$

where $a_{0},...,a_{n}$ are differentiable functions from $\mathbb{R} \rightarrow \mathbb{R}$ and $f:\mathbb{R} \rightarrow \mathbb{R}$\\

Show that $V$ has dimention $n$

\subsection{Alternating sum of harmonic series}
Show that $\sum_{n=1}^{\infty}\frac{(-1)^{n+1}}{n}$ converges to $\ln(2)$

\end{document}
